%% LyX 2.3.6 created this file.  For more info, see http://www.lyx.org/.
%% Do not edit unless you really know what you are doing.
\documentclass[english]{article}
\usepackage[T1]{fontenc}
\usepackage[latin9]{luainputenc}

\makeatletter
%%%%%%%%%%%%%%%%%%%%%%%%%%%%%% User specified LaTeX commands.
\usepackage{babel}


\makeatother

\usepackage{babel}
\begin{document}
\title{Element perfomance design and specification}
\title{Requirements }
\maketitle

\section*{Mechanical }
\begin{enumerate}
\item Shall have a method of navigating through the field 
\item Shall have at least one pusher/puller hereby referred to as a pusher
for moving blocks around the field 
\item Shall fit in a 8in by 8in box at start of play and 18in cylinder during
all of play 
\item Shall have cad files made for each component 
\item Shall be able to move each pusher with at least a minimum of x m/s
in any direction 
\item Shall have a weight that is less than 5 lbs 
\end{enumerate}

\section*{Electrical }
\begin{enumerate}
\item Shall be able to sense items in front of the robot 
\item Shall be able to detect which side of the field each pusher is on 
\item Shall be able to detect the state of the pusher 
\item Shall be able to detect the velocity or absolute position of all wheels 
\item Shall have a motor controller to regulate current direction and amount
sent to each actuator 
\item Shall have electrical designs for circuit 
\end{enumerate}

\section*{Software }
\begin{enumerate}
\item Shall be able to interrupt when a pusher has entered the other side
of field 
\item Shall have a looping algorithm to move blocks to either side of the
field 
\item Shall be able to move the the robot autonomously across the field 
\item Shall be able to choose which side of the field blocks should be moved
to 
\item Shall use a robotic simulation environment to test the code 
\item Shall detect current state of pusher at initialization and move to
a designated initialization state for pusher Robot 
\item shall have a designated initialization state 
\item Shall use github to version control the software 
\end{enumerate}

\section*{Safety and logistics }
\begin{enumerate}
\item Shall have a hard reset button that can be easily reached to turn
off all components of the robot 
\item Shall have a software reset button that returns the robot to initialization
state 
\item Shall have a active budget of 40 dollars and fabrication budget of
300 dollars 
\end{enumerate}

\end{document}
